\documentclass[12pt,titlepage]{article}

\usepackage[a4paper,top=30mm,bottom=30mm,left=20mm,right=20mm]{geometry}
\usepackage[utf8]{inputenc}
\usepackage{xspace}
\usepackage{listings}
\usepackage{optionman}
\usepackage{url}
\usepackage{booktabs}
\usepackage{xcolor}
%\usepackage[binary-units]{siunitx}
%\usepackage{TheSansUHH}

\newcommand{\Gdiff}{\texttt{genomediff}\xspace}
\newcommand{\RGdiff}{\texttt{run\_genomediff.rb}\xspace}
\newcommand{\GenomeTools}{\textit{GenomeTools}\xspace}
\newcommand{\Gt}{\texttt{gt}\xspace}
\newcommand{\Kr}{\ensuremath{K_r}\xspace}
\newcommand{\Gtsuffixerator}{\texttt{gt suffixerator}\xspace}
\newcommand{\Gtpackedindex}{\texttt{gt packedindex mkindex}\xspace}
\newcommand{\ESA}{ESA\xspace}
\newcommand{\FastA}{FastA\xspace}
\newcommand{\File}[1]{\texttt{\small #1}}
\newcommand{\ShuS}{\textit{shustrings}\xspace}

\lstset{language=bash,
basicstyle=\ttfamily
}
\definecolor{darkgreen}{rgb}{0.3,0.5,0.3}
\definecolor{darkblue}{rgb}{0.3,0.3,0.5}
\definecolor{darkred}{rgb}{0.5,0.3,0.3}
\lstdefinelanguage{LUA}{%
sensitive=true,%
columns=fixed,%
keywordstyle=[1]{\color{darkblue}\bfseries},%
keywordstyle=[2]{\color{darkgreen}\bfseries},%
morekeywords=[1]{local,if,then,else,end,while,do, coroutine,yield},% Official LUA keywords
morekeywords=[2]{units},% Your private keywords
otherkeywords={.,=,~,*,>,:},%
morestring=[b]",%
stringstyle={\color{darkred}\itshape},%
breaklines=true,%
linewidth=\textwidth,%
comment=[l]{--}%
}
\title{\Gdiff user manual}

\author{\begin{tabular}{c}
  \textit{Dirk Willrodt}\\[1cm]
  Research Group for Genome Informatics\\
  Center for Bioinformatics\\
  University of Hamburg\\
  Bundesstrasse 43\\
  20146 Hamburg\\
  Germany\\[1cm]
  \url{willrodt@zbh.uni-hamburg.de}\\
\end{tabular}}

\begin{document}
%\tsuhhfamily
\maketitle

\section*{This Manual}
Some text is highlighted by different fonts according to the following rules.

\begin{itemize}
\item \texttt{Typewriter font} is used for the names of software tools.
\item \File{Small typewriter font} is used for file names.
\item \begin{footnotesize}\texttt{Footnote sized typewriter font}
      \end{footnotesize} with a leading
      \begin{footnotesize}\texttt{'-'}\end{footnotesize}
      is used for program options.
\item \Showoptionarg{small italic font} is used for the argument(s) of an
      option.
\end{itemize}


\section{Introduction}

This document describes \Gdiff, a software tool for measuring evolutionary
distances between sets of closely related genomes. These distances are Jukes-Cantor
corrected divergence between the pairs of genomes, that is the number of
mutations per base between them.

This distance is called \Kr and is based on so called \ShuS
\cite{HAU:DOM:WIE:2008,HAU:PFA:DOM:WIE:2009,HAU:REE:PFA:2011}. The calculation
of all pairwise distances is alignment free, but the resulting distances have
the same biological meaning as if calculated with a multiple sequence alignment.

This software is only able to process closely related distances, because \Kr is
only reliable for distances $<0.5$.

\Gdiff is written in C and it is based on the \GenomeTools
library~\cite{genometools}. It is called as part of the single binary named \Gt.

The source code can be compiled on 32-bit and 64-bit platforms without making
any changes to the sources.

\section{Building \Gdiff} \label{Building}

As \Gdiff is part of the \GenomeTools software suite, a source distribution of
\GenomeTools must be obtained, e.g.\@ via the \GenomeTools home page
(\url{http://genometools.org}), and decompressed into a source directory:

\begin{lstlisting}
$ tar -xzvf genometools-X.X.X.tar.gz
$ cd genometools-X.X.X
\end{lstlisting}

Then, it suffices to call \lstinline{make} to compile the source using the provided
makefile.

It is recommended to use the 64bit-version of the \GenomeTools executable if
Your system supports this.

The option \lstinline!amalgamation=yes! allows the compiler to use better
optimization.

\begin{lstlisting}
$ make 64bit=yes amalgamation=yes
\end{lstlisting}

After successful compilation, the \GenomeTools executable containing \Gdiff
can then be installed for system-wide use as follows (do this as root):

\begin{lstlisting}
$ make install
\end{lstlisting}%$

If a \texttt{prefix=<path>} option is appended to this line, a custom directory
can be specified as the installation target directory, e.g.\@

\begin{lstlisting}
$ make install prefix=/home/user/gt
\end{lstlisting}%$
%$

will install the \Gt binary in the \File{/home/user/gt/bin} directory. Please
also consult the \File{README} and \File{INSTALL} files in the root
directory of the uncompressed source tree for more information and
troubleshooting advice.

\section{Usage}

\subsection{\Gdiff command line options}

Since \Gdiff is part of \GenomeTools it is invoked as follows:

\texttt{gt genomediff [\textit{\small options}]}
\Showoption{pck}$|$\Showoption{esa} \Showoptionarg{indexname}

where \Showoptionarg{indexname} is the path without file extension of either
an enhanced suffix array (\ESA), indicated by \Showoption{esa}, or a FM-index
which would require option
\Showoption{pck}.

A short description of all possible options is given in Table \ref{tab:gdopts}

\begin{table}[hbpt]
  \centering
  \caption{\Gdiff{} command line options}
\begin{footnotesize}
  \label{tab:gdopts}
  \begin{tabular}{lp{0.6\textwidth}}
    \toprule
    \Showoptiongroup{Input options}
    \Showoption{esa} \Showoptionarg{indexname} & specify path to enhanced
    suffix array \\
    \Showoption{pck} \Showoptionarg{indexname} & specify path to FM-index \\
    \Showoption{unitfile} \Showoptionarg{filename} & specify file which
    groups the inputfiles into units to be compared to each other. See
    Listing \ref{code:lua} for an example.
    \\\midrule
    \Showoptiongroup{\ESA options}
    \Showoption{scan} \Showoptionarg{yes$|$no} & scan \ESA sequentially or load
    into memory. \Showoptionarg{yes} is the default and recommended.
    \\\midrule
    \Showoptiongroup{Miscellaneous options}
    \Showoption{v} & be verbose \\
    \Showoption{help} & print help and exit \\
    \Showoption{help+} & print extended help and exit \\\bottomrule
  \end{tabular}
\end{footnotesize}
\end{table}

\begin{lstlisting}[%
  float=hbpt,%
  showlines=true,%
  frame=tb,%
  caption={Example of lua-unitfile; {\bfseries\small The section
    '\texttt{units}' is mandatory, '\texttt{genome1/2}' are examples of
    names, filenames are paths without filename extension.}},%
  label={code:lua}, language=LUA]
units = {
  genome1 = { "file1", "file2" },
  genome2 = { "file3", "file4" }
}
\end{lstlisting}

\subsection{Input files}

The indexfiles can be built from sequence files using \Gtsuffixerator{} for
enhanced suffix arrays or \Gtpackedindex{} for FM-indices. The use of
enhanced suffix arrays is recommended.

\subsection{Output}

The output on \texttt{stdout} consists of a line with the number of genomes or units
that were compared. It is followed by a quadratic matrix of pairwise
distances where each line consists of a file or unitname and tabulator
separated distance values.

Depending on the options of the \Gt call there are one ore more comment lines
which start with \texttt{\#}.

\section{Simple ruby script}

\GenomeTools comes with a collection of helpful scripts, mainly written in
Ruby. These scripts can be found in the \File{scripts} folder in the
\GenomeTools source.

To simplify calling \Gdiff there exists a script called \RGdiff. It calls all
necessary tools with reasonable standard parameters. It also prepares the
data, by removing all stretches of wildcards and replacing them with single
\texttt{'N'}s.

\subsection{\RGdiff command line options}

The script is called as follows:

\texttt{(PATH)/\RGdiff [\textit{\small options}]} \Showoptionarg{files}

Where \Showoptionarg{files} are the genomes to be compared in separate
\FastA files.

A short description of all possible options is given in Table
\ref{tab:scopts}.

\begin{table}[hbpt]
  \centering
  \caption{\RGdiff command line options}
\begin{footnotesize}
  \label{tab:scopts}
  \begin{tabular}{lp{0.6\textwidth}}
    \toprule
    \Showoptiongroup{Index options}
    \Showoption{-pck} & use (packed) fm-index, default is \ESA\\
    \Showoption{p}/\Showoption{-parts} \Showoptionarg{N} & affects index
    calculation, splits index into \Showoptionarg{N} parts, thereby reducing
    memory requirements.\\
    \Showoption{m}/\Showoption{-maxmem} \Showoptionarg{M} & as \Showoption{-parts} but
    \Showoptionarg{M} is an upper bound for memory consumption in
    MiB during index calculation and number of parts is
    calculated.\\
    \Showoption{-name} \Showoptionarg{NAME} & the path/basename for the index,
    defaults to \Showoptionarg{esa} or \Showoptionarg{pck}.
    \\\midrule
    \Showoptiongroup{Data options}
    \Showoption{-reduceN} & Scans the sequence files before index
    construction and reduces all wildcard stretches to single
    \texttt{'N'}s\\
    \Showoption{-unitfile} \Showoptionarg{unitfile} & unitfile, groups files
    into genomes, see Listing \ref{code:lua} for an example.
    \\\midrule
    \Showoptiongroup{Advanced options}
    \Showoption{-idxopts} \Showoptionarg{'opts'} & options for
      indexconstruction, see help of \Gtsuffixerator or \Gtpackedindex for
      options. Use \texttt{''} to group.\\
    \Showoption{-diffopts} \Showoptionarg{'opts'} & options for
    \Kr-calculation. Use \texttt{''} to group.
    \\\midrule
    \Showoptiongroup{Miscellaneous options}
    \Showoption{h}/\Showoption{-help} & print help and exit
    \\\bottomrule
  \end{tabular}
\end{footnotesize}
\end{table}

\section{Example}

This section describes two example scenarios, first comparing multiple
genomes that are organised in separate multiple \FastA-files and the second
comparing two genomes consisting of multiple files each.

\subsection{Compare genomes in separate files}

Assume whe have three files \File{genome1.fas}, \File{genome2.fas} and
\File{genome3.fas} each of which could contain multiple \FastA entries. Our
machine has only 2\,GiB RAM. Assuming the index construction
would need 5\,GiB, we need to split it in at least three parts or
restrict maximal memory requirements.

The simplest way to calculate the distance matrix for these three genomes
would be to call:

\lstinline{run_genomediff.rb --parts 3 genome1.fas genome2.fas genome3.fas}

or:

\lstinline{run_genomediff.rb --memlimit 1024 genome1.fas genome2.fas genome3.fas}

This would use \Gtsuffixerator to build an \ESA of the three sequence files
including the reverse complement. \Showoption{-memlimit} should be reasonable
less than available main memory.

The index will be stored in the current folder as separate files with
basename \File{esa} in the current directory and can be used for further
experiments.

The distance matrix will be printed to the terminal, in order to store it use

\lstinline{run_genomediff.rb genome1.fas genome2.fas genome3.fas > outfile}

to store the results in file \File{outfile}. This file might look like this:
\begin{verbatim}
# Table of Kr
3
genome1.fas	0.000000	0.115125	0.267473
genome2.fas	0.115125	0.000000	0.293082
genome3.fas	0.267473	0.293082	0.000000
\end{verbatim}

This tabulator separated table can be used with \textit{Phylip} or
\textit{R} to calculate a phylogenetic tree.

Another way to calculate the same distances would be to call the tools
separately, for example create an \ESA whith \Gtsuffixerator like this:

\begin{lstlisting}
gt suffixerator                         \
-mirrored -dna -suf -tis -lcp -ssp      \
-db genome1.fas genome2.fas genome3.fas \
-indexname esa                          \
-memlimit 1024                          \
\end{lstlisting}

and afterwards call \Gdiff:

\begin{lstlisting}
gt genomediff -esa esa > outfile
\end{lstlisting}

\subsection{Compare two genomes in multiple files}

Assume we have two genomes that consist of multiple chromosomes in separate
files. For example genome1 consists of \File{g1\_chr1.fas} and
\File{g1\_chr2.fas} while the two files for genome2 are named accordingly.
The unitfile should be saved like this:
\begin{lstlisting}[language=lua]
units = {
  genome1 = { "g1_chr1", "g1_chr2" },
  genome2 = { "g2_chr1", "g2_chr2" }
}
\end{lstlisting}
The name of the unitfile in our example will be \File{units}

Now we could call \RGdiff like this:

\begin{lstlisting}
run_genomediff.rb --unitfile units \
g1_chr1.fas g1_chr2.fas g2_chr1.fas g2_chr2.fas
\end{lstlisting}

\begin{verbatim}
# Table of Kr
2
genome1	0.000000	0.115125
genome2	0.115125	0.000000
\end{verbatim}

\section*{Bibliography}
\bibliographystyle{unsrt}
\bibliography{gtmanuals}
\end{document}
% vim:spell spelllang=en_gb
